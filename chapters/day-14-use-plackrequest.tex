\chapter{Use Plack::Request}\label{day-14-use-plackrequest}

Plack is not a framework per se, but is more of a toolkit that contains
PSGI server implementations as well as utilities like
\program{plackup} (\autoref{day-3-using-plackup}),
\module{Plack::Test} (\autoref{day-13-use-placktest-to-test-your-application})
and middleware components (\autoref{day-10-using-plack-middleware}).

Since Plack project is a revolution from
\href{http://search.cpan.org/perldoc?HTTP::Engine}{\module{HTTP::Engine}}, there
seems a demand to write a quick web application in request/response
style handler API. \module{Plack::Request} gives you a nice Object Oriented API
around PSGI environment hash and response array, just like Rack's
\module{Rack::Request} and Response objects. It could also be used as a library
when writing a new middleware component, and a base class for
requests/responses when you write a new web application framework based
on Plack.

\section{Use Plack::Request and
Response}\label{use-plackrequest-and-response}

\module{Plack::Request} is a wrapper around PSGI environment, and the code goes
like this:

\inputperl{day14-1.pl}

The only thing you need to change, if you're migrating from
\module{HTTP::Engine}, is the first line of the application to create a
\module{Plack::Request} out of PSGI env (\lstinline!shift!) and then call
\lstinline!finalize! to get an array reference out of Response object.

Many other methods like \lstinline!path_info!, \lstinline!uri!,
\lstinline!param!, \lstinline!redirect! etc. work like
\module{HTTP::Engine::Request} and Response object which is very similar to
\href{http://search.cpan.org/dist/Catalyst-Runtime}{Catalyst}'s Request
and Response object.

\section{Plack::Request and Plack}\label{plackrequest-and-plack}

\module{Plack::Request} is available as part of Plack on CPAN. Your framework can
use \module{Plack::Request} to handle parameters and can also make it run on
other PSGI server implementations such as mod\_psgi.

\section{Use Plack::Request or not?}\label{use-plackrequest-or-not}

Directly using \module{Plack::Request} in the \file{.psgi} code is quite
handy to quickly write and test your code but not really recommended for
a large scale application. It's exactly like writing a thousand lines of
\file{.cgi} script where you could factor out the application code
into a module (\file{.pm} files). The same thing applies to
\file{.psgi} file: it's best to create an application class by
using and possibly extending \module{Plack::Request}, and then have just a few
lines of code in \file{.psgi} file with \module{Plack::Builder}
to configure middleware components (\autoref{day-11-using-plackbuilder}).

\module{Plack::Request} is also supposed to be used from a web application
framework to adapt to PSGI interface (\autoref{day-8-adapting-web-frameworks-to-psgi}).

