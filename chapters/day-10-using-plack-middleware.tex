\chapter{Using Plack middleware}\label{day-10-using-plack-middleware}

\section{Middleware}\label{middleware}

Middleware is a concept in PSGI (as always, stolen from Python's WSGI
and Ruby's Rack) where we define components that plays the both side of
a server and an application.

\begin{figure}[htbp]
\centering
\includesvg[width=8cm]{images/pylons_as_onion}
\caption{WSGI middleware onion 
  \emph{(Image courtesy of Pylons project for Python WSGI)}}
\end{figure}

This picture illustrates the middleware concept very well. The PSGI
application is in the core of the Onion layers, and middleware
components wrap the original application in return, and they preprocess
as a request comes in (outer to inner) and then postprocess the response
as a response goes out (inner to outer).

Lots of functionalities can be added to the PSGI application by wrapping
it with a middleware component, from HTTP authentication, capturing
errors to logging output or wrapping JSON output with JSONP.

\section{Plack::Middleware}\label{plackmiddleware}

\href{http://search.cpan.org/perldoc?Plack::Middleware}{\module{Plack::Middleware}}
is a base class for middleware components and it allows you to write
middleware really simply but in a reusable fashion.

Using Middleware components written with \module{Plack::Middleware} is easy, just
wrap the original application with \lstinline!wrap! method:

\inputperl{day10-1.pl}

This example wraps the original application with StackTrace middleware
(which is actually enabled
\href{http://advent.plackperl.org/2009/12/day-3-using-plackup.html}{by
default using \program{plackup}}) with the \lstinline!wrap! method. So when the
wrapped application throws an error, the middleware component catches
the error to
\href{http://bulknews.typepad.com/blog/2009/10/develstacktraceashtml.html}{display
a beautiful HTML page} using \module{Devel::StackTrace::AsHTML}.

Some other middleware components take parameters, in which case you can
pass the parameters as a hash after \lstinline!$app!, like:

\inputperl{day10-2.pl}

Installing multiple middleware components is tedious especially since
you need to \lstinline!use! those modules first, and we have a quick
solution for that using a DSL style syntax.

\inputperl{day10-3.pl}

We'll see more about \module{Plack::Builder} tomorrow.

\section{Middleware and Frameworks}\label{middleware-and-frameworks}

The beauty of Middleware is that it can wrap \emph{any} PSGI
application. It might not be obvious from the code examples, but the
wrapped application can be anything, which means you can
\href{http://advent.plackperl.org/2009/12/day-7-use-web-application-framework-in-psgi.html}{run
your existing web application in the PSGI mode} and apply middleware
components to it. For instance, with \module{CGI::Application}:

\inputperl{day10-4.pl}

This will enable the Basic authentication middleware to \module{CGI::Ap|pli|ca|tion}
based application. You can do the same with
\href{http://plackperl.org/\#frameworks}{any other frameworks that
supports PSGI}.

