\chapter{Write your own middleware}
\label{day-23-write-your-own-middleware}

Let's finish up this middleware discovery with ``Do It Yourself''
tutorial now.

\section{Writing middleware}\label{writing-middleware}

PSGI middleware behaves like a normal PSGI application but wraps the
original PSGI application, so from the server it looks like an
application but from an application it looks like a server (plays both
sides).

A simple middleware that fakes HTTP user-agent would be like this:

\inputperl{day23-1.pl}

The app would display ``Hello iPhone'' only if a request comes with
iPhone browser (\emph{Mobile Safari}), but the middleware adds that
phrase to all incoming requests, so if you run this application and open
the page with any browsers, you'll always see ``Hello iPhone''. And the
default Access Log would say:

\begin{shell}
127.0.0.1 - - [23/Dec/2009 12:34:31] "GET / HTTP/1.1" 200 12 "-" "Mozilla/5.0 
(Macintosh; U; Intel Mac OS X 10_6_2; en-us) AppleWebKit/531.21.8 (KHTML, like
Gecko) Version/4.0.4 Safari/531.21.10 (Mobile Safari)"
\end{shell}

You can see \lstinline!" (Mobile Safari)"! is added to the tail of User-Agent
string.

\section{Make it a reusable middleware}\label{make-it-a-reusable-middleware}

So that was a good example of writing your own middleware in
\file{.psgi}. If it is one-time middleware that you can quickly
whip up then that's great, but you often want to make it generic enough
or reusable in other applications too. Then you should use
\module{Plack::Middleware}.

\inputperl{day23-2.pl}

That's it. All you have to do is to inherit from \module{Plack::Middleware} and
defines options that your middleware would take, and implement
\lstinline!call! method that would delegate to \lstinline!$self->app!
which is a wrapped application. This middleware is now compatible to
\href{http://advent.plackperl.org/2009/12/day-11-using-plackbuilder.html}{\module{Plack::Builder}
DSL} so you can say:

\inputperl{day23-3.pl}
%
to fake all incoming requests as it comes with the good old Internet
Explorer, and you can also use \lstinline!enable_if! to
\href{http://advent.plackperl.org/2009/12/day-18-load-middleware-conditionally.html}{conditionally
enable} this middleware.

\section{Post process requests}\label{post-process-requests}

The previous examples does pre-processing of PSGI request
\lstinline!$env! hash, what to do about the response? It's almost the
same:

\inputperl{day23-4.pl}

This is an \emph{evil} middleware component that changes all the status
code to 500 unless it's 200 OK. Not sure if there is any use for this
but it's simple enough for a quick example.

Because some servers implement special
\href{http://bulknews.typepad.com/blog/2009/10/psgiplack-streaming-is-now-complete.html}{streaming
interface} to delay HTTP response, this middleware doesn't really work
with such an interface. Dealing with this special callback interface in
individual middleware is not efficient, so we have a special callback
interface in \module{Plack::Middleware} to make this easy:

\inputperl{day23-5.pl}

Pass the response \lstinline!$res! to \lstinline!response_cb! and set
the callback to wrap the real response, and the method takes care of the
direct response and delayed response.

\section{Namespaces}\label{namespaces}

In this example we use \module{Plack::Middleware} namespace to make middleware,
but it doesn't really have to be. If you think your middleware is
generic enough for all PSGI apps can benefit, feel free to use the
namespace, but if the middleware is too specific for your own needs, or
works only with a particular application framework, then use whatever
namespace, like:

\inputperl{day23-6.pl}
%
and then use the + (plus) sign to indicate the fully qualified
namespace,

\inputperl{day23-7.pl}
%
or use the non-DSL API,

\inputperl{day23-8.pl}
%
and they should work just fine.

