\chapter{Reloading applications}\label{day-4-reloading-applications}

Yesterday I introduced the basics of \program{plackup} and its command line
options. Today I'll show you more!

\section{Reload the application as necessary}
\label{reload-the-application-as-necessary}

During development you often change your Perl code, saved in
\file{.psgi} or \file{.pm} files. Because the Plack server
launched by the \program{plackup} command is a persistent process you need to
restart your server whenever your code changes. This is a little
painful.

So there's an option to watch for changes to files under your working
directory and reload the application as needed: \switch{-r} (or
\switch{{-}{-}reload}).

\begin{shell}
$ plackup -r hello.psgi
\end{shell}

It will watch files under the current directory by default, but you can
change it to watch additional locations by using the \switch{-R}
option (note the uppercase).

\begin{shell}
$ plackup -R lib,/path/to/scripts hello.psgi
\end{shell}

As you can see, multiple paths can be monitored by combining them with
a comma.

By default \program{plackup} uses a dumb timer to scan the whole directory, but if
you're on Linux and have \module{Linux::Inotify2} installed or on Mac OS and have
\module{Mac::FSEvents} installed filesystem notification is used so it's more
efficient.

\section{Server auto-detection vs -r}\label{r-vs-server-auto-detection}

In Day 3 I told you that \program{plackup}'s automatic server detection is smart
enough to tell if PSGI application uses an event module such as
\module{AnyEvent} or \module{Coro} and choose the correct backend. Be aware that this
automatic selection doesn't work if you use the \switch{-r} option
because \program{plackup} uses a delayed loading technique to reload apps in
forked processes. It's recommended that you explicitly set the server
with the \switch{-s} option when using the \switch{-r} option.

\section{Reloading sucks? Shotgun!}\label{reloading-sucks-shotgun}

Reloading a module or application in a persistent Perl process can cause
problems. For instance, module package variables could be redefined or
overwritten and then get stuck in a bad state.

Plack now has the Shotgun loader, inspired by
\href{http://github.com/rtomayko/shotgun}{Rack's Shotgun}, which solves
the reloading problem by loading the app on \emph{every request} in a
forked child environment.

Using the Shotgun loader is easy:

\begin{shell}
$ plackup -L Shotgun myapp.psgi
\end{shell}

This will delay the compilation of your application until runtime. When a
request is received it will fork off a new child process to compile your
app and return the PSGI response over the pipe. You can also preload
modules in the parent process that are not likely to be updated to
reduce the time needed to compile your application.

For instance, if your application uses \module{Moose} and \module{DBIx::Class} then use
the following options:

\begin{shell}
$ plackup -MMoose -MDBIx::Class -L Shotgun myapp.psgi
\end{shell}

This speeds up the time required to compile your application at runtime.


