\chapter{Lint your application and middleware}
\label{day-21-lint-your-application-and-middleware}

We've been talking about adapting existing web frameworks to PSGI
(\autoref{day-8-adapting-web-frameworks-to-psgi})
and writing a new application using
PSGI as an interface, but we haven't talked about error handling.

\section{Handling errors}\label{handling-errors}

We have an awesome stack trace (\autoref{day-3-using-plackup})
middleware enabled by default, so if an end user
application throws an error, we can catch them and display a nice error
page. But what if there is an error or a bug in one of middleware
components, or web application framework adapters themselves?

Try this code:

\begin{shell}
$ plackup -e 'sub { return [ 0, {"Content-Type","text/html"}, "Hello" ] }'
\end{shell}

Again, writing a raw PSGI interface is not something end users would do
every day, but this could be a good emulation of what would happen if
there's a bug in one of middleware components or framework adapters
itself.

When you access this application with the browser, the server dies with:

\begin{shell}
Not an ARRAY reference at lib/Plack/Util.pm line 145.
\end{shell}
%
or something similar. This is because the response format is invalid per
the PSGI interface: the status code is not valid, HTTP headers are not an
array ref but a hash reference and the response body is a string instead
of an array ref.

\section{Lint middleware}\label{lint-middleware}

Checking them in the individual server for every request at runtime
is \emph{possible} but not \emph{ideal}: that will be a duplicate of
codes, and doing so in every request is not efficient from the
performance standpoint. We should better validate if an application,
middleware or server backend conforms to the PSGI interface using the
test suite during the development and disable that when running on
production for the best performance.

\module{Middleware::Lint} is the middleware to validate request and response
interface. Run the application above with the middleware:

\begin{shell}
$ plackup -e 'enable "Lint"; sub { return [ 0, { "Content-Type"=>"text/html" }, ["Hello"] ] }'
\end{shell}

and now requests for the application would give a nice stack trace
saying:

\begin{shell}
status code needs to be an integer greater than or equal to 100 at ...
\end{shell}
%
since now the Lint middleware checks if the response in the valid PSGI
format.

When you develop a new framework adapter or a middleware component, be
sure to check with \module{Middleware::Lint} during the development.

\section{Writing a new PSGI server}\label{writing-a-new-psgi-server}

\module{Middleware::Lint} validates both request and response interface, so this
can be used when you develop a new PSGI web server as well. However if
you are a server developer there's a more comprehensive testing tool to
make sure your server behaves correctly, and that is \module{Plack::Test::Suite}.

You can look at the existing tests in the \lstinline!t/Plack-Handler!
directory for how to use this utility, but it defines lots of expected
requests and responses pairs to test a new PSGI server backend. Existing
\module{Plack::Han|dler} backends included in Plack core distribution as well as
other CPAN distributions all pass this test suite.

