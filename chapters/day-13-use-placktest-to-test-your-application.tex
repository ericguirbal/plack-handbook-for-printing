\chapter{Use Plack::Test to test your application}
\label{day-13-use-placktest-to-test-your-application}

\section{Testing}\label{testing}

There are many ways to test web applications, either with a live server
or using a mock request technique. Some web application frameworks allow
you to write an unit test using one of those methods, but the way you
write tests differs per framework of your choice.

\module{Plack::Test} gives you \emph{a unified interface} to test \emph{any} web
applications and frameworks that is compatible to PSGI using \emph{both}
mock request and live HTTP server.

\section{Using Plack::Test}\label{using-placktest}

Using \module{Plack::Test} is pretty simple and it's of course compatible to the
Perl's standard testing protocol
\href{http://testanything.org/wiki/}{TAP} and
\href{http://search.cpan.org/perldoc?Test::More}{\module{Test::More}}.

\inputperl{day13-1.pl}

Create or load PSGI application like usual (you can use
\href{http://search.cpan.org/perldoc?Plack::Util}{\module{Plack::Util}}'s
\lstinline!load_psgi! function if you want to load an app from a
\file{.psgi} file), and call \lstinline!test_psgi! function to test
the application. The second argument is a callback that acts as a
testing client.

You can use the named parameters as well, like the following.

\inputperl{day13-2.pl}

The client code takes a callback (\lstinline!$cb!), which you can pass
an \module{HTTP::Request} object that would return \module{HTTP::Response} object, like
normal \module{LWP::UserAgent} would do, and you can make as many requests as you
want, and test various attributes and response details.

Save that code as \file{.t} file and use the tool such as
\program{prove} to run the tests.

\section{Use HTTP::Request::Common}\label{use-httprequestcommon}

This is not required, but recommended to use
\href{http://search.cpan.org/perldoc?HTTP::Request::Common}{\module{HTTP::Request::Common}}
when you want to make an HTTP request, since it's more obvious and less
code to write:

\inputperl{day13-3.pl}

Notice that you can even omit the scheme and hostname, which would
default to \url{http://localhost/} anyway.

\section{Run in a server/mock mode}\label{run-in-a-servermock-mode}

By default the \lstinline!test_psgi! function's callback runs as a
\emph{Mock HTTP} request mode, turning a \module{HTTP::Request} object into a
PSGI env hash and then run the PSGI application, and returns the
response as a \module{HTTP::Response} object.

You can change this to live HTTP mode, by setting either the package
variable \lstinline!$Plack::Test::Impl! or the environment variable
\lstinline!PLACK_TEST_IMPL! to the string \lstinline!Server!.

\inputperl{day13-4.pl}

By using the environment variable, you don't really need to change the
\file{.t} code:

\begin{shell}
$ env PLACK_TEST_IMPL=Server prove -l t/test.t
\end{shell}

This will run the PSGI application using the Standalone server backend
and uses \module{LWP::UserAgent} to send the live HTTP request. You don't need to
modify your testing client code, and the callback would automatically
adjust host names and port numbers depending on the test configuration.

\section{Test your web application framework with
Plack::Test}\label{test-your-web-application-framework-with-placktest}

Once again, the beauty of PSGI and Plack is that everything written to
run for the PSGI interface can be used for \emph{any} web application
frameworks that speaks PSGI. By running your web application framework 
in PSGI mode (\autoref{day-7-use-web-application-framework-in-psgi}), 
you can also use \module{Plack::Test}:

\inputperl{day13-5.pl}

You can of course do the same thing against any frameworks that supports
PSGI.

