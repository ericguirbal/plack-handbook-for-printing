\chapter{Maps multiple apps with mount and URLMap}
\label{day-12-maps-multiple-apps-with-mount-and-urlmap}

\section{Hello World! but anyone
else?}\label{hello-world-but-anyone-else}

Throughout the advent calendar we most of the time use the simplest web
application using the ``Hello World'' example, like

\inputperl{day12-1.pl}
%
what about more complex examples, like you have multiple applications,
each of which inherit from one of the web application frameworks, and
use one of apache magic like mod\_alias etc.

\section{Plack::App::URLMap}\label{plackappurlmap}

\module{Plack::App::URLMap} allows you to \emph{composite} multiple PSGI
applications into one application, to dispatch requests to multiple
applications using the URL path, or even with virtual host based
dispatch.

\inputperl{day12-2.pl}

So you have two apps, one is to say hi to John and another to Bob, and
say if you want to run this two application on the same server. With
\module{Plack::App::URLMap}, you can do this.

\inputperl{day12-3.pl}

There you go. Your app now dispatches all requests beginning with
\lstinline!/john! to \lstinline!$app1! which says ``Hello John'' and
\lstinline!/bob! to \lstinline!$app2!, which is to say ``Hello Bob''. As
a result, all requests to unmapped paths, like the root (\url{/}) gives
you 404.

The environment variables such as \lstinline!PATH_INFO! and
\lstinline!SCRIPT_NAME! are automatically adjusted so it just works like
when your application is mounted using Apache's mod\_alias or CGI
scripts. Your application framework should always use
\lstinline!PATH_INFO! to dispatch requests, and concatenate with
\lstinline!SCRIPT_NAME! to build links.

\section{mount in DSL}\label{mount-in-dsl}

This \lstinline!mount! interface with \module{Plack::App::URLMap} is quite
useful, so we decided to add to \module{Plack::Builder} DSL itself, which is
again an inspiration by \module{Rack::Builder}, using the syntax
\lstinline!mount!:

\inputperl{day12-4.pl}

Requests to \url{/john} is handled exactly the same way with the normal
URLMap. But this example uses \lstinline!builder! for \url{/bob}, so it
enables the basic authentication to display the ``Hello Bob'' page. This
should be syntactically equivalent to:

\inputperl{day12-5.pl}
%
but obviously, with less code to write and more obvious to understand
what's going on.

\section{Multi tenant frameworks}\label{multi-tenant-frameworks}

Of course you can use this URLMap and mount API to run multiple
framework applications on one server. Imagine you have three
applications, ``Foo'' which is based on Catalyst, ``Bar'' which is based
on \module{CGI::Application} and ``Baz'' which is based on Squatting. Do this:

\inputperl{day12-6.pl}

And now you have three applications, each of which inherit from
different web framework, running on the same server (via \program{plackup} or
other \module{Plack::Handler::*} implementations) mapped on different paths.

