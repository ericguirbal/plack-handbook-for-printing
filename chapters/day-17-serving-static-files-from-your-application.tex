\chapter{Serving static files from your application}
\label{day-17-serving-static-files-from-your-application}

On
\href{http://advent.plackperl.org/2009/12/day-5-run-a-static-file-web-server-with-plack.html}{day
5} we talked about serving files from the current directory using
\program{plackup}. Now that we've learned how to
\href{http://advent.plackperl.org/2009/12/day-10-using-plack-middleware.html}{use
middleware} and
\href{http://advent.plackperl.org/2009/12/day-12-maps-multiple-apps-with-mount-and-urlmap.html}{compound
multiple applications with URLMap} it's extremely trivial to add a
functionality you definitely need when developing an application:
serving static files.

\section{Serving files from a certain
path}\label{serving-files-from-a-certain-path}

Most frameworks come with this feature but with PSGI and Plack,
frameworks don't need to implement this feature anymore. Just use the
Static middleware.

\inputperl{day17-1.pl}

This will intercept all requests beginning with \url{/static} and map that
to the root directory \file{htdocs}. So requests to
\url{/static/images/foo.jpg} will result in serving a file
\url{./htdocs/static/images/foo.jpg}.

Often you want to overlap or cofigure the directory names, so a request
to the URL \url{/static/index.css} mapped to \file{./static-files/index.css},
here's how to do that:

\inputperl{day17-2.pl}

The important thing here is to use a callback and a pattern match
\lstinline!sub { s/// }! instead of a plain regular expression
(\lstinline!qr!). The callback is tested against a request path and if
it matches, the value of \lstinline!$_! is being used as a request path.
So in this example we tested to see if the request begins with
\url{/static/} and in that case, strip off that part, and map the files
under \file{./static-files/}.

As a result, \url{/static/foo.jpg} would become
\file{./static-files/foo.jpg}. All requests not matching the pattern match
just pass through to the original \lstinline!$app!.

\section{Do it your own with URLMap and
App::File}\label{do-it-your-own-with-urlmap-and-appfile}

Just like Perl there's more than one way to do it. When you grok how to
use
\href{http://advent.plackperl.org/2009/12/day-12-maps-multiple-apps-with-mount-and-urlmap.html}{mount
and URLMap} then using \module{App::File} with mount should be more intuitive.
The previous example can be written like this:

\inputperl{day17-3.pl}

Your mileage may vary, but I think this one is more obvious. Static's
callback based configuration allows you to write more complex regular
expression, which you can't do with URLMap and mount, so choose
whichever fits your need.

