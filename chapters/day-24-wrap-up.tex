\chapter{Wrap up}\label{day-24-wrap-up}

24 days have passed so fast and this is the last entry for this Plack
advent calendar.

\section{Best Practices}\label{best-practices}

Plack and PSGI are still really young projects but we've already
discovered a couple of suggestions and advices to write a new PSGI
application or a framework.

When you write a new framework, be sure to have an access to the PSGI
environment hash from end users applications or plugin developers,
either directly or with an accessor method. This allows your framework
to share and extend functionality with middleware components like Debug
or Session.

Do not write your application logic in \file{.psgi} files using
\module{Plack::Request}. It's like writing a 1{,}000 lines of CGI script using
\module{CGI.pm}, so if you think that's your favorite i won't give you any
further advice, but usually you want to make your application
\href{http://advent.plackperl.org/2009/12/day-13-use-placktest-to-test-your-application.html}{testable}
and reusable by making it a class or an object. Then your
\lstinline!.psgi! code is just a few lines of code to create a PSGI
application out of it and apply some middleware components.

Think twice before using \module{Plack::App::*} namespace. \module{Plack::App} namespace
is for middleware components that do not act as a \emph{wrapper} but
rather an \emph{endpoint}. Proxy, File, Cascade and URLMap are the good
examples. If you write a blog application using Plack, \textbf{Never}
call it \module{Plack::App::Blog}, okay? Name your software by what it does, not
how it's written.

\section{Explore more stuff}\label{explore-more-stuff}

Most of the Plack gangs use \href{http://github.com/}{GitHub} for the
source control and
\href{http://github.com/search?langOverride=\&q=plack\&repo=\&start_value=1\&type=Repositories}{searching
for repositories with ``Plack''} would give you a fresh look of what
would look like an interesting idea. You can also search for modules on
CPAN with
\href{http://search.cpan.org/search?query=plack\&mode=module}{Plack} or
\href{http://search.cpan.org/search?query=psgi\&mode=module}{PSGI}. I
keep track of good blog posts and stuff on delicious, so you can see
them tagged with \href{http://delicious.com/miyagawa/psgi}{psgi} or
\href{http://delicious.com/miyagawa/plack}{Plack}.

\section{Getting in touch with the dev
team}\label{getting-in-touch-with-the-dev-team}

Again, Plack is a fairly young project. It's just been 3 months since we
gave this project a birth. There are many things that could get more
improvements, so if you come across one of them, don't stop there. Let
us know what you think is a problem, give us an insight how it could be
improved, or if you're impatient, fork the project on GitHub and send us
patches.

We're chatting on IRC channel \#plack on irc.perl.org and there's a
\href{http://groups.google.com/group/psgi-plack}{mailing list} and
\href{http://github.com/plack/Plack/issues}{an issue tracker on GitHub}
to communicate with us.

\section{On a final note\ldots{}}\label{on-a-final-note}

It's been an interesting experiment of writing 24 articles for 24 days,
and I'm glad that I finished this myself. Next year, i'm looking forward
to having your own advent entries to make the community based advent
calendar.

I wish you a Very Merry Christmas and a Happy New Year.
