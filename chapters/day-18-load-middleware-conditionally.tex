\chapter{Load middleware conditionally}
\label{day-18-load-middleware-conditionally}

I've introduced a couple of middleware components. Some of them are
useful and could be enabled globally, while others might be better
enabled on certain conditions. Today we'll talk about a solution to
this.

\section{Load middleware
conditionally}\label{load-middleware-conditionally}

Conditional middleware is a super (or meta) middleware that takes one
middleware and enable that middleware based on a runtime condition.
Let's take some examples:

\begin{itemize}
\itemsep1pt\parskip0pt\parsep0pt
\item
  You want to enable
  \href{http://advent.plackperl.org/2009/12/day-16-adding-jsonp-support-to-your-app.html}{JSONP
  middleware} only if the path begins with \file{/public}.
\item
  You don't want to enable
  \href{http://advent.plackperl.org/2009/12/day-15-authenticate-your-app-with-middleware.html}{Basic
  Auth} if the request comes from local IP.
\end{itemize}

We investigated how they deal with situations like this in WSGI and
Rack, but couldn't find a generic solution, and they mostly just
implement options to individual component, which did not look cool for
me.

\section{Middleware::Conditional}\label{middlewareconditional}

The Conditional middleware is an ultimate flexible solution to this:

\inputperl{day18-1.pl}

We added a new keyword to \module{Plack::Builder} \lstinline!enable_if!, which
takes a block that gets evaluated in the request time (\lstinline!$_[0]!
there is the \lstinline!$env! hash) and if the block returns true, run
the wrapped application but otherwise pass through.

This example code examines if the request comes from a local network and
runs a basic authentication otherwise.

Conditional is implemented as a normal piece of middleware, and
internally this is equivalent to:

\inputperl{day18-2.pl}

But it's a little boring to write, so we added a DSL version, which I
recommend to use :)

