\chapter{Using Plack::Builder}\label{day-11-using-plackbuilder}

\href{http://advent.plackperl.org/2009/12/day-10-using-plack-middleware.html}{Yesterday}
we saw how to enable Plack middleware components in \file{.psgi} file, using
its \lstinline!wrap! class method. The way you \lstinline!use! the
middleware and then wrap the \lstinline!$app! with \lstinline!wrap! is
tedious and not intuitive, so we have a DSL (Domain Specific Language)
to make it much easier, and that is \module{Plack::Builder}.

\section{Using Plack::Builder}\label{using-plackbuilder}

The way you use \module{Plack::Builder} is so easy. Just use the keywords
\lstinline!builder! and \lstinline!enable!:

\inputperl{day11-1.pl}

This takes the original application (\lstinline!$app!) and wraps it with
Deflater, \module{Auth::Basic} and JSONP middleware components (inner to outer).
So it's equivalent to:

\inputperl{day11-2.pl}
%
but without lots of \lstinline!use!ing the module which is anti DRY.

\section{Outer to Inner, Top to the
bottom}\label{outer-to-inner-top-to-the-bottom}

Notice that the order of middleware wrapping is in reverse? The
builder/enable DSL allows you to \emph{wrap} application so the line
close to the original \lstinline!$app! is \emph{inner}, while the first
one in the top is \emph{outer}. You can compare that with
\href{http://pylonshq.com/docs/en/0.9.7/_images/pylons_as_onion.png}{the
onion picture} and see that it's more obvious: something closer to the
application is inner.

The keyword \lstinline!enable! takes the middleware name without the
\module{Plack::Middleware::} prefix but in case you want to enable some other
namespace, like \module{My|Fra|me|work::PSGI::MW::Foo}, you can say:

\inputperl{day11-3.pl}

The key here is to use the plus (+) sign to indicate that it is a fully
qualified class name.

\section{What's happening behind}\label{whats-happening-behind}

If you're curious what \module{Plack::Builder} is doing, take a look at the code
and see what's happening. The \lstinline!builder! takes the code block
and executes the code, and take the result (return value of the last
statement) as an original application (\lstinline!$app!), and then
returns the wrapped application by applying Middleware in the reverse
order. So it is important to have \lstinline!$app! in the last line
inside the \lstinline!builder! block, and have the \lstinline!builder!
statement as the final statement in \file{.psgi} file as well.

\section{Thanks, Rack}\label{thanks-rack}

This idea of \module{Plack::Builder} is totally an inspiration by
\href{http://m.onkey.org/2008/11/18/ruby-on-rack-2-rack-builder}{\module{Rack::Builder}}.
You can see that they use the \lstinline!use! keyword but obviously we
can's \emph{use} that in Perl :) so we chose \lstinline!enable! instead.
You can see that they have \lstinline!map! which is to map applications
to a different path, and we'll talk about its equivalent in Plack
tomorrow ;)

