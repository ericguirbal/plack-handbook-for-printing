\chapter{Using plackup}\label{day-3-using-plackup}

In the
\href{http://advent.plackperl.org/2009/12/day-2-hello-world.html}{Day 2}
article I used the \program{plackup} command to run the Hello World PSGI
application.

The script \program{plackup} is a command line launcher of PSGI applications inspired by
Rack's rackup command. It can be used to run any PSGI application saved
in a \file{.psgi} file with one of the PSGI web server backends using Plack
handlers. The usage is simple; just pass a \file{.psgi} file path to the
command:

\begin{shell}
$ plackup hello.psgi
HTTP::Server::PSGI: Accepting connections at http://0:5000/
\end{shell}

You can actually omit the filename if you're trying to run the file
named \file{app.psgi} in the current directory.

The default backend is chosen using one of the following methods:

\begin{itemize}
\itemsep1pt\parskip0pt\parsep0pt
\item
  If the environment variable \lstinline!PLACK_SERVER! is set it is
  used.
\item
  If some environment specific variable like
  \lstinline!GATEWAY_INTERFACE! or \lstinline!FCGI_ROLE! is set the
  backend for CGI or FCGI is used accordingly.
\item
  If the loaded \file{.psgi} file uses a specific event module like
  \module{Any|Event}, \module{Co|ro} or \module{POE} the equivalent and most appropriate backend is
  chosen automatically.
\item
  Otherwise, fallback to the default ``Standalone'' backend implemented
  in the \module{HTTP::Ser|ver::PSGI} module.
\end{itemize}

You can also specify the backend yourself from the command line using
the \switch{-s} (or \switch{{-}{-}server}) switch:

\begin{shell}
$ plackup -s Starman hello.psgi
\end{shell}

By default the \program{plackup} command enables three middleware components to
aid development: \module{Lint}, \module{AccessLog}, and \module{StackTrace}. You can disable them
with the \switch{-E} (or \switch{{-}{-}environment}) switch:

\begin{shell}
$ plackup -E production -s Starman hello.psgi
\end{shell}

In the case that you really want to use the \lstinline!development!
Plack environment but want to disable the default middleware there is
the \switch{{-}{-}no-default-middleware} option.

Other command line switches can be passed to the server. You can specify
the server listen port with:

\begin{shell}
$ plackup -s Starlet --host 127.0.0.1 --port 8080 hello.psgi
Plack::Handler::Starlet: Accepting connections at http://127.0.0.1:8080/
\end{shell}
%
or specify the unix domain socket the FCGI backend should listen on
with:

\begin{shell}
$ plackup -s FCGI --listen /tmp/fcgi.sock app.psgi
\end{shell}

For more options for \program{plackup}, run \lstinline!perldoc plackup! from the
command line. You'll see more \program{plackup} options and hacks tomorrow as
well.


