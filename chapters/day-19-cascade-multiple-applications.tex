\chapter{Cascade multiple applications}
\label{day-19-cascade-multiple-applications}

Conditional middleware (\autoref{day-18-load-middleware-conditionally}) and
URLMap app (\autoref{day-12-maps-multiple-apps-with-mount-and-urlmap})
have something in common: they're PSGI applications but both take
PSGI application or middleware and dispatch them. This is the beauty of
PSGI application and middleware architecture and today's application is
another example of this.

\section{Cascading multiple
applications}\label{cascading-multiple-applications}

Cascading can be useful if you have a couple of applications and runs in
order, then try until it returns a successful response. This is
sometimes called
\href{http://en.wikipedia.org/wiki/Chain-of-responsibility_pattern}{Chain
of responsibility} design pattern and often used in web applications
such as
\href{http://perl.apache.org/docs/2.0/user/handlers/intro.html}{mod\_perl
handlers}.

\section{Cascade Application}\label{cascade-application}

\module{Plack::App::Cascade} allows you to compound multiple applications and
tries them to return the first non-404 response.

\inputperl{day19-1.pl}

This application is mapped to \file{/static} using URLMap, and all
requests will try the three directories specified in \lstinline!@paths!
using \module{App::File} application and returns the first found file. It might
be useful if you want to serve static files but want to cascade from
multiple directories like this.

\section{Cascade different apps}\label{cascade-different-apps}

\inputperl{day19-2.pl}

This will create two applications, one with Catalyst and the other with
\module{CGI::Application} and runs two applications in order. Suppose you have an
overlapping URL structure and \file{/what/ever.cat} served with the
Catalyst application and \file{what/ever.cgiapp} served with the
\module{CGI::Application} app.

Well that might sound crazy and I guess it's better to use URLMap to map
two applications in different paths, but if you \emph{really want} to
cascade them, this is the way to go :)

