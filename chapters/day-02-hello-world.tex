\chapter{Hello World}\label{day-2-hello-world}

The first program you write in any programming language is the one that
prints ``Hello World''. Let's follow that tradition for PSGI as well.

\begin{note}
  Today's code is written to a raw PSGI interface to help
you understand what's going on. In reality you should never have to do
this unless you're a web application framework developer. Instead you
should use one of the \href{http://plackperl.org/\#frameworks}{existing
frameworks that supports PSGI}.
\end{note}

\section{Hello, World}\label{hello-world}

Here's the minimal code that prints ``Hello World'' to the client:

\inputperl{day2-1.pl}

A PSGI application is a Perl subroutine reference (a coderef) and is
usually referenced as \lstinline!$app! (it could be named anything
obviously). It takes exactly one argument \lstinline!$env! (which is not
used in this code) and returns an array ref containing status, headers,
and body. That's it.

Save this code in a file named \file{hello.psgi} and then use the
\program{plackup} command to run it:

\begin{shell}
$ plackup hello.psgi
HTTP::Server::PSGI: Accepting connections at http://0:5000/
\end{shell}

The \program{plackup} command runs your application with the default HTTP server
\module{HTTP::Ser|ver::PSGI} on localhost port 5000. Open the URL
\url{http://127.0.0.1:5000/} and you should see the ``Hello World'' page.

\section{Give me something
different}\label{give-me-something-different}

The ``Hello World'' program is the simplest code imaginable. We can do more here. Let's
read and display the client information using the PSGI environment hash:

\inputperl{day2-2.pl}

This code will display the remote address using the PSGI environment
hash. It will normally be 127.0.0.1 if you're running the server on your
localhost. The PSGI environment hash contains lots of information about
an HTTP connection like incoming HTTP headers and request paths, much
like the CGI environment variables.

Want to display something that isn't just text? We can do this by
reading a file:

\inputperl{day2-3.pl}

This app would serve \file{favicon.ico} if the request path looks like
\url{/favicon.ico}, the ``Hello World'' page for requests to the root (\url{/}) and
otherwise a~404. You can see that a Perl filehandle (\lstinline!$fh!) is
a valid PSGI response, and you can use any valid HTTP status code for a
response.


