\chapter{Running CGI scripts on Plack}\label{day-9-running-cgi-scripts-on-plack}

For the last couple of days we've been talking about how to convert
existing CGI based applications to PSGI, and then run them as a PSGI
application. Today we will show you the ultimate way to run \emph{any}
CGI scripts as a PSGI application, most of the time unmodified.

\href{http://search.cpan.org/perldoc?CGI::PSGI}{\module{CGI::PSGI}} is a subclass
of \module{CGI.pm} to allow you a very easy migration from \module{CGI.pm} with only
\emph{a few lines of code changes} to run it on PSGI environment. But
what about a messy or legacy CGI script that just prints to \lstinline!STDOUT! a lot
and is not easy to fix?

\href{http://search.cpan.org/perldoc?CGI::Emulate::PSGI}{\module{CGI::Emulate::PSGI}}
is a module to run any CGI based perl program in a PSGI environment.
Whatever messy/old script that prints stuff to \lstinline!STDOUT! or directly reads
HTTP headers from \lstinline!%ENV! would just work because that's what
\module{CGI::Emulate::PSGI} tries to emulate. The original POD of
\module{CGI::Emulate::PSGI} was illustrating it like:

\inputperl{day9-1.pl}
%
to run existing CGI application that may or may not use \module{CGI.pm} (\module{CGI.pm}
caches lots of environment variables so it needs
\lstinline!initialize_globals()! call to clear out the previous request
variables).

A few days ago on my flight from San Francisco to London to attend
London Perl Workshop I was hacking on something more intelligent, that
is to take any CGI scripts and compiles it into a subroutine. The module
is named
\href{http://search.cpan.org/perldoc?CGI::Compile}{\module{CGI::Compile}} and
should be best used combined with \module{CGI::Emulate::PSGI}.

\inputperl{day9-2.pl}

There's also
\href{http://search.cpan.org/perldoc?Plack::App::CGIBin}{\module{Plack::App::CGIBin}}
Plack application to run existing CGI scripts written in Perl as PSGI
applications, suppose you have bunch of CGI scripts in
\file{/path/to/cgi-bin}, you'll run the server with:

\begin{shell}
$ plackup -MPlack::App::CGIBin -e 'Plack::App::CGIBin->new(root => "/path/to/cgi-bin"))'
\end{shell}

And that will mount the path \url{/path/to/cgi-bin}, so suppose
you have \file{foo.pl} in that directory, you can access
\url{http://localhost:5000/foo.pl} to run the CGI application as a PSGI over
the \program{plackup}, just like the scripts running on Apache mod\_perl Registry
mechanism.

