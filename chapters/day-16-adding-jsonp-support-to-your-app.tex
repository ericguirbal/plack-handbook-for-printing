\chapter{Adding JSONP support to your app}
\label{day-16-adding-jsonp-support-to-your-app}

Today we'll see another very simple but useful example of a middleware
component, this time to add functionality beyond just basic HTTP
functions.

\section{JSONP}\label{jsonp}

\href{http://ajaxian.com/archives/jsonp-json-with-padding}{JSONP}
(JSON-Padding) is a technology to wrap JSON in a JavaScript callback
function. This is normally useful when you want to allow your JSON-based
content included programatically in the third party websites using HTML
\lstinline!script! tags.

\section{Middleware::JSONP}\label{middlewarejsonp}

Assume your web application returns a JSON encoded data with the
Content-Type \lstinline!application/json!, again with a simple inline
PSGI application:

\inputperl{day16-1.pl}

Adding a JSONP support is easy using \module{Middleware::JSONP}:

\inputperl{day16-2.pl}

So it's just one line! The middleware checks if the response content
type is \lstinline!application/json! and if so, checks if there is a
\lstinline!callback! parameter in the URL. So a request to
\url{/whatever.json} continues to return the JSON but requests to
\url{/whatever.json?callback=myCallback} would return:

\inputperl{day16-3.pl}
%
with the Content-Type \lstinline!text/javascript!. Content-Length is
automatically adjusted if there's any.

\section{Works with frameworks}\label{works-with-frameworks}

Supporting JSONP in addition to JSON would be fairly trivial for most
frameworks to do, but \module{Middleware::JSONP} should be an example of the
things that could be done in Plack middleware layer with no complexity.

And of course, this JSONP middleware should work with any existing web
frameworks that emits JSON output. So with Catalyst:

\inputperl{day16-4.pl}

And then using \module{Catalyst::Engine::PSGI} and \module{Plack::Builder}, you can add a
JSONP support to this controller.

\inputperl{day16-5.pl}

Accidentally this
\href{http://search.cpan.org/perldoc?Catalyst::View::JSON}{\module{Catalyst::View::JSON}}
is my module :) and supports JSONP callback configuration by default,
but there is more than one way to do it!

