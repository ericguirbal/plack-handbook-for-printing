\chapter{Adapting web frameworks to PSGI}\label{day-8-adapting-web-frameworks-to-psgi}

The biggest benefit of PSGI for web application framework developers is
that once you adapt your framework to run on PSGI you can forget and
throw away everything else that you needed to deal with to, say, handle
the differences between a bunch of FastCGI servers.

Similarly, if you have a large scale web application, open source or
proprietary, you probably have your own web application framework (or a
base class or the like).

Today's entry discusses how to convert existing web application
frameworks to the PSGI interface.

\section{CGI.pm based framework}\label{cgi.pm-based-framework}

In Day 7 we saw how to run a \module{CGI::Application} based application in PSGI
using \module{CGI::Application::PSGI}. \module{CGI::Application}, as the name suggests,
uses \module{CGI.pm}, so using \module{CGI::PSGI} instead and defining a new runner class
is the easiest way to go.

\inputperl{day8-1.pl}

That's quite simple, isn't it? \module{CGI::Application}'s \lstinline!run()!
method usually returns the whole output, including HTTP headers and
content body. As you can see, the module has a gross hack to disable the
header generation since you can use the \lstinline!psgi_header! method
of \module{CGI::PSGI} to generate the status code and HTTP headers as an array
ref.

I've implemented PSGI adapters for
\href{http://search.cpan.org/perldoc?HTML::Mason}{Mason} and
\href{http://search.cpan.org/perldoc?Maypole}{Maypole} and the code
pretty much all looked alike:

\begin{itemize}
\itemsep1pt\parskip0pt\parsep0pt
\item
  Create \module{CGI::PSGI} out of \lstinline!$env! and set that instead of the
  default \module{CGI.pm} instance.
\item
  Disable HTTP header generation if needed.
\item
  Run the app main dispatcher.
\item
  Extract the HTTP headers to be sent, use \lstinline!psgi_header! to
  generate the status and headers.
\item
  Extract the response body (content).
\end{itemize}

\section{Adapter based framework}\label{adapter-based-framework}

If the framework in question already uses adapter based approaches to
abstract server environments it should be much easier to adapt to PSGI
by reusing most of the CGI adapter code. Here's the code to adapt
\href{http://search.cpan.org/perldoc?Squatting}{Squatting} to PSGI.
Squatting uses the \module{Squatting::On::*} namespace to adapt to environments
like mod\_perl, FastCGI, or even other frameworks like Catalyst or
\module{HTTP::Engine}. It was extremely easy to write
\href{http://search.cpan.org/perldoc?Squatting::On::PSGI}{\module{Squatting::On::PSGI}}:

\inputperl{day8-2.pl}

That's very straightforward, especially when compared with
\href{http://cpansearch.perl.org/src/BEPPU/Squatting-0.70/lib/Squatting/On/CGI.pm}{\module{Squat|ting::On::CGI}}.
It's almost a line-by-line copy (with some adjustment) using
\module{Plack::Request} to parse parameters instead of \module{CGI.pm}.

Similarly, Catalyst uses the \module{Catalyst::Engine} abstraction and
\href{http://search.cpan.org/perldoc?Catalyst::Engine::PSGI}{\module{Ca|ta|lyst::En|gi|ne::PSGI}}
is the adapter to run Catalyst on PSGI, where most of the code is copied
from CGI.

\section{mod\_perl centric
frameworks}\label{modux5fperl-centric-frameworks}

Some frameworks are centered around mod\_perl's API, in which case we
can't use the approaches we've seen here. Instead, you should probably
start by mocking \module{Apache::Request} APIs using a fake/mock object. Patric
Donelan, a WebGUI developer, explains his approach to make a
mod\_perl-like API in
\href{http://blog.patspam.com/2009/plack-roundup-at-sf-pm}{his blog
post}. The
\href{http://github.com/pdonelan/webgui/blob/plebgui/lib/WebGUI/Session/Plack.pm}{mock
request class linked} is a good place to start.

