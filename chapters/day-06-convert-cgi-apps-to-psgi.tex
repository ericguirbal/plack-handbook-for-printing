\chapter{Convert CGI apps to PSGI}\label{day-6-convert-cgi-apps-to-psgi}

The most popular web server environments to run web applications for
Perl have been CGI, FastCGI, and mod\_perl. \module{CGI.pm} is one of the Perl
core modules that happens to run fine in any of these environments (with
some tweaks). This means many web applications and frameworks use \module{CGI.pm}
to deal with environment differences because it's the easiest.

\href{http://search.cpan.org/perldoc?CGI::PSGI}{\module{CGI::PSGI}} is a CGI
module subclass that makes it easy to migrate existing \module{CGI.pm} based
applications to PSGI. Imagine you have the following CGI application:

\inputperl{day6-1.pl}

This is a very simple CGI application and converting this to PSGI is
easy using the \module{CGI::PSGI} module:

\inputperl{day6-2.pl}

\lstinline!CGI::PSGI->new($env)! takes the PSGI environment hash and
creates an instance of \module{CGI::PSGI}, which is a subclass of \module{CGI.pm}. All
methods including \lstinline!param()!, \lstinline!query_string!, etc. do
the right thing to get the values from the PSGI environment rather than
CGI's ENV values.

The \lstinline!psgi_header! utility method works just like CGI's
\lstinline!header! method and returns the status code and an array
reference containing the list of HTTP headers.

Tomorrow I'll talk about how to convert existing web frameworks that use
\module{CGI.pm} to PSGI using \module{CGI::PSGI}.


