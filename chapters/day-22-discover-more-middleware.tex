\chapter{Discover more middleware}
\label{day-22-discover-more-middleware}

Christmas is coming near and there aren't enough days to explore more
middleware components. Today I'll show you a quick intro of great
middleware components that I haven't had time to show.

\section{ErrorDocument}\label{errordocument}

When you die out from an application or display some ``Forbidden'' error
message when an auth wasn't successful you'll probably want to display a
custom error page based on the response status code. ErrorDocument is
exactly the middleware that does this, like Apache's ErrorDocument
directive.

\inputperl{day22-1.pl}

You can just map arbitrary error code to a static file path to be
served. You can enable StackTrace middleware during the development and
then this ErrorDocument middleware on the production so as to display
nicer error pages.

This middleware is included in the Plack core distribution.

\section{Session}\label{session}

Actually this is (again) a steal from
\href{http://rack.rubyforge.org/}{Rack}. Rack defines
\lstinline!rack.session! as a standard Rack environment hash and defines
the interface as Ruby's built-in Hash object. We didn't define it as
part of the standard interface but stole the idea and actual
implementation a lot.

\inputperl{day22-2.pl}

By default Session will save the session in on-memory hash, which
wouldn't work with the prefork (or multi process) servers. It's shipped
with a couple of default store engines such as
\href{http://search.cpan.org/perldoc?CHI}{CHI}, so it's so easy to adapt
to other storage engines, exactly like we see with other middleware
components such as Auth.

Session data is stored as a plain hash reference in
\lstinline!psgix.session! key in the PSGI env hash. Application and
frameworks with access to PSGI env hash can use this Session freely in
the app by wrapping it with \module{Plack::Session} module, like in Tatsumaki:

\inputperl{day22-3.pl}

And the nice thing is that \emph{any} PSGI apps can share this session
data as long as they use the same storage etc. Some existing framework
adapters don't have an access to this environment hash from end users
application yet, so it should be updated gradually in the near future.

Session middleware is developed by Stevan Little on
\href{http://github.com/stevan/plack-middleware-session}{GitHub} and is
available on CPAN as well.

\section{Debug}\label{debug}

This is a steal from \href{http://github.com/brynary/rack-bug}{Rack-bug}
and \href{http://github.com/robhudson/django-debug-toolbar}{django debug
toolbar}. By enabling this middleware you'll see the handy debug
``panels'' in the right side where you can click and see the detailed
data and analysis about the request.

The panels include Timer (the request time), Memory (how is memory
increased if there's any leaks), Request (Detailed request headers) and
Responses (Response headers etc.) and so on.

\inputperl{day22-4.pl}

Using it is so easy as this, and you an also pass the list of
\lstinline!panels! to enable only certain panels or additional non
default panels.

More extensions for the panels, such as DBI query profiler or Catalyst
log dumper are being developed on
\href{http://github.com/miyagawa/plack-middleware-debug/}{GitHub}.

\section{Proxy}\label{proxy}

It's often useful to proxy HTTP requests to another application, either
running on the internet or inside the same network. The former would be
necessary if you want to proxy long poll or some JSON API from your
application that doesn't support JSONP (because of Cross domain origin
policy), and the latter would be to run applications on different
machine and use your app as a reverse proxy, though chances are you want
to use frontend web servers like nginx, lighttpd or perlbal to do the
job.

Anyway, \module{Plack::App::Proxy} is the middleware to do this:

\inputperl{day22-5.pl}

Proxy middleware is developed by Lee Aylward on
\href{http://github.com/leedo/Plack-App-Proxy}{GitHub}.

\section{More}\label{more}

There are more middleware components available in the Plack
distribution, and on
\href{http://search.cpan.org/search?query=plack+middleware\&mode=dist}{CPAN}.
Not all middleware components are supposed to be great, but certainly
they can be shared and used by most frameworks that support PSGI.

