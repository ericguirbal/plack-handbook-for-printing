\chapter{Access your local app from the internet}
\label{day-20-access-your-local-app-from-the-internet}

\begin{note}
  The ReverseHTTP service we mention here is not available as of 2012.
\end{note}

These days laptops with modern operation systems allows you to quickly
develop a web application and test it locally with its local IP address.
Often you want to test your application with a global access, to show
off your work to friends who don't have an access to your local network,
or you're writing a web application that works as a
\href{http://www.webhooks.org/}{webhooks} callback.

\section{Reverse HTTP to the
rescue}\label{reverse-http-to-the-rescue}

There are many solutions to this problem, but one notable solution is
\href{http://www.reversehttp.net/}{ReverseHTTP}. It is a very simple
specification of client-server-gateway protocol that uses pure HTTP/1.1
payloads, and what's nice about it is that there's a demo gateway
service running on reversehttp.net, so you can actually use it for demo
or testing purpose pretty quickly without setting up servers etc.

If you're curious how this really works, take a look at
\href{http://www.reversehttp.net/specs.html}{the spec}. The reason why
it's called \emph{Reverse} HTTP is that your application (server) acts
as a long-poll HTTP client and the gateway server sends back an HTTP
request as a response. This might sound complex but well, it's really
simple :)

\section{Plack::Server::ReverseHTTP}\label{plackserverreversehttp}

\href{http://search.cpan.org/~miyagawa/Plack-Server-ReverseHTTP-0.01/}{\module{Plack::Server::ReverseHTTP}}
is a Plack server backend that implements this ReverseHTTP protocol, so
your PSGI based application can be accessed from the outside world via
this reversehttp.net gateway service.

To use ReverseHTTP, install the required modules and run this:

\begin{shell}
$ plackup -s ReverseHTTP -o yourhostname --token password -e 'sub { [200, ["Content-Type","text/plain"], ["Hello"]] }'
Public Application URL: http://yourhostname.www.reversehttp.net/
\end{shell}

The \switch{-o} switch is an alias for \switch{{-}{-}host} for \program{plackup} (because
\switch{-h} is taken for \switch{{-}{-}help} :)), and you should
specify the subdomain (label) you're going to use. You should also
supply \switch{{-}{-}token} which is like a generic password so nobody
else can use your label once registered. You can omit this option if you
\emph{really} want anyone else to take that subdomain over.

The console will display the address (URL) like seen, and open the URL
from the browser and viola! You see the ``Hello'' page, right?

\section{Use with frameworks}\label{use-with-frameworks}

Of course because this is a PSGI server backend, you can use with
\emph{any} frameworks. Want to use it with Catalyst application?

\begin{shell}
$ catalyst.pl MyApp
$ cd MyApp
$ ./scripts/myapp_create.pl PSGI
$ plackup -o yourhost --token password ./scripts/myapp.psgi
\end{shell}

That's it! The default Catalyst application will now be accessible with
the URL \url{http://yourhost.reversehttp.net/} from anywhere in the world.

\section{Notes}\label{notes}

ReverseHTTP.net gateway service is an experimental service and there's
no SLA or whatever, so I don't really think it's usable for production
environment and such. But it's really handy and useful to quickly test
your application that needs a global access, or show off your work to
friends that don't have an internal access. Much easier than other
solutions that require other software like SSH or VPN tunneling.

