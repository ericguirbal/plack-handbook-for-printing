\chapter{Authenticate your app with Middleware}
\label{day-15-authenticate-your-app-with-middleware}

There're lots of Plack middleware components out there, whether in Plack
core distribution as well as separate distributions on CPAN. While I've
been writing this Plack advent calendar lots of people have shown their
interest and taken ideas out of my wishlist.

From today we'll introduce some of the nice middleware components that
you can use quickly to enhance any of your PSGI ready applications.

\section{Basic authentication}\label{basic-authentication}

Since Plack middleware wraps an application, the best thing it can do is
pre-process or post-process to do things around HTTP layers. Today let's
talk about the Basic Authentication.

Adding a basic authentication can be done in multiple ways: you can do
that in the web application framework layer if it's supported in your
framework. In case of Catalyst it's
\href{http://search.cpan.org/perldoc?Catalyst::Authentication::Credential::HTTP}{\module{Catalyst::Authentication::Credential::HTTP}}.
Just like other Catalyst tools, it allows you to configure the
authentication from simple to very complex, by using credential (how to
authenticate: basic and/or digest) and store (how to authorize username
and passwords).

Otherwise you can do the authentication in the web server layer. For
instance if you run your application with Apache and mod\_perl, using
Apache's default mod\_auth module to authenticate is pretty easy and
handy for development, while it limits the ability to share ``how to
authenticate users'' since usually you need to write your custom module
to do things like database backed authentication.

Plack middleware allows web application frameworks to share such a
functionality, mostly with a pretty simple Perl callback system, and
\module{Plack::Middleware::Auth::Basic} is to do this for Basic authentication.
And this is why most Plack standalone servers do not have an
authentication system: it's best implemented as a middleware component.

\section{Using
Plack::Middleware::Auth::Basic}\label{using-plackmiddlewareauthbasic}

Just like
\href{http://advent.plackperl.org/2009/12/day-10-using-plack-middleware.html}{other
middleware}, using \module{Auth::Basic} middleware is quite simple:

\inputperl{day15-1.pl}

This adds a basic authentication to your application \lstinline!$app!,
and the user \emph{admin} can sign in with the password \emph{foobar}
and nobody else. The successful signed-in user gets
\lstinline!REMOTE_USER! set in PSGI \lstinline!$env! hash so it can be
used in the applications and is logged using the standard AccessLog
middleware.

Since it's a callback based, adding another authentication system such
as Kerberos would be pretty trivial and easy with modules such as
\module{Authen::Simple}:

\inputperl{day15-2.pl}

The same way you can use lots of
\href{http://search.cpan.org/search?query=authen+simple\&mode=all}{\module{Authen::Simple}
backends} with small changes.

\section{With URLMap}\label{with-urlmap}

URLMap allows you to compound multiple apps into one app, so combined
with Auth middleware, you can run the same application in a auth
vs.~non-auth mode, using different paths:

\inputperl{day15-3.pl}

This way you run the same \lstinline!$app! in \url{/public} and
\url{/private} paths, while \url{/private} requires a basic authentication
and \url{/public} doesn't. (Inlining \lstinline!$env->{REMOTE_USER}!, or
whatever application logic in \file{.psgi} is not really recommended -- I just
used it to explain it in an obvious way)

