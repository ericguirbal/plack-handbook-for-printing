\chapter{Run a static web server with Plack}\label{day-5-run-a-static-web-server-with-plack}

The Plack distribution comes with some ready made PSGI applications in
the \module{Plack::App} namespace. Some of them are pretty handy, for instance
\href{http://search.cpan.org/perldoc?Plack::App::File}{\module{Plack::App::File}}
and
\href{http://search.cpan.org/perldoc?Plack::App::Directory}{\module{Plack::App::Directory}}.

\module{Plack::App::File} translates a request path like
\url{/foo/bar.html} into a local file like
\url{/path/to/htdocs/foo/bar.html}, opens the file handle, and
passes it back as a PSGI response. It basically does what a static web
server like lighttpd, nginx or Apache does.

\module{Plack::App::Directory} is a wrapper around \module{Plack::App::File} that gives a
directory index, just like
\href{http://httpd.apache.org/docs/2.0/mod/mod_autoindex.html}{Apache's
mod\_autoindex} does.

Using these applications is easy. Just write a \file{.psgi} file like this:

\inputperl{day5-1.pl}
%
and run it with \program{plackup}:

\begin{shell}
$ plackup file.psgi
\end{shell}

Now you can access any file under your \url{~/public_html} with
the URL \url{http://localhost:5000/somefile.html}

You can also use \module{Plack::App::Directory}. This time let's run it with just
the \program{plackup} command without a \file{.psgi} file:

\begin{shell}
$ plackup -MPlack::App::Directory -e 'Plack::App::Directory->new(root => "$ENV{HOME}/Sites")'
HTTP::Server::PSGI: Accepting connections at http://0:5000/
\end{shell}

The \program{plackup} command, like the perl command, accepts flags like
\switch{-I} (include path), \switch{-M} (modules to load), and
\switch{-e} (the code to eval), so it's easy to load these
\module{Plack::App::*} applications without ever touching a \file{.psgi} file!

There are a couple of other \module{Plack::App} applications in the Plack
distribution.


