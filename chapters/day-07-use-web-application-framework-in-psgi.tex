\chapter{Use web application framework in PSGI}\label{day-7-use-web-application-framework-in-psgi}

Since we started the Plack and PSGI project in September 2009 there has
been a lot of feedback from the authors of popular frameworks such as
Catalyst, Jifty, and \module{CGI::Application}.

\href{http://cgi-app.org/}{\module{CGI::Application}} is one of the most
``traditional'' CGI-based web application framework and it uses \module{CGI.pm}
exclusively to handle web server environments just like we discussed
yesterday.

Mark Stosberg, the current maintainer of \module{CGI::Application}, and I have
been collaborating on adding PSGI support to \module{CGI::Application}. We
thought of multiple approaches including adding native PSGI support to
\module{CGI.pm}, but we ended up implementing
\href{http://search.cpan.org/perldoc?CGI::PSGI}{\module{CGI::PSGI}} as a \module{CGI.pm}
wrapper and then using
\href{http://search.cpan.org/perldoc?CGI::Application::PSGI}{\module{CGI::Application::PSGI}}
to run existing \module{CGI::Application} code \emph{unmodified} in a PSGI
compatible mode.

All you have to do is install \module{CGI::Application::PSGI} from CPAN and write
a \file{.psgi} file that looks like this:

\inputperl{day7-1.pl}

Then use \program{plackup} (\autoref{day-3-using-plackup})
to run the application with a standalone server or any of the other
backends.

Similarly, most web frameworks that support PSGI provide a plugin,
engine, or adapter to make the framework run in PSGI mode. For instance,
\href{http://www.catalystframework.org/}{Catalyst} has a
\module{Catalyst::Engine::*} web server abstraction and
\href{http://search.cpan.org/perldoc?Catalyst::Engine::PSGI}{\module{Catalyst::Engine::PSGI}}
is the engine to adapt Catalyst to run on PSGI. 

\begin{note}
  As of Catalyst 5.8 released in 2011, Catalyst supports PSGI by default and
  there's no need to install a separate engine.
\end{note}

The point is that with support from web frameworks you often won't need
to modify a single line of code in your application to use PSGI. And by
switching to PSGI there are lots of benefits like being able to use the
toolchain of \program{plackup}, \module{Plack::Test}, and middleware which we'll discuss in
future advent entries.

