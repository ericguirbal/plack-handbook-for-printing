\chapter{Getting Plack}\label{day-1-getting-plack}

The most important step to get started is to install
\href{http://search.cpan.org/dist/Plack}{Plack} and other utilities.
Because PSGI and Plack are just normal Perl module distributions the
installation is easy: just launch your CPAN shell and type:

\begin{shell}
cpan> install PSGI Plack
\end{shell}

\href{http://search.cpan.org/dist/PSGI}{PSGI} is a specification
document for the PSGI interface. By installing the distribution you can
read the documents in your shell with the \lstinline!perldoc PSGI! or
\lstinline!perldoc PSGI::FAQ! commands. Plack gives you the standard
server implementations, core middleware components, and utilities like
\program{plackup} and \module{Plack::Test}.

Plack doesn't depend on any non-core XS modules so with any Perl
distribution later than 5.8.1 (which was released more than 6 years
ago!) it can be installed very easily, even on platforms like Win32 or
Mac OS X without developer tools (i.e.~C compilers).

If you're a developer of web applications or frameworks (I suppose you
are!), it's highly recommended you install the optional module bundle
\href{http://search.cpan.org/dist/Task-Plack}{\module{Task::Plack}} as well. The
installation is as easy as typing:

\begin{shell}
cpan> install Task::Plack
\end{shell}

You will be prompted with a couple of questions depending on your
environment. If you're unsure whether you should or should not install,
just type return to select the default. You'll get optional XS speedups
by default, while other options like non-blocking environments are
disabled by default.

Start reading the documentation with \lstinline!perldoc Plack! to get
prepared.


